\documentclass[10pt]{article}  % smaller base font
\usepackage{graphicx}
\usepackage{geometry}
\geometry{margin=0.8in}  % narrower margins

\usepackage{setspace}
\setstretch{0.95}  % slightly tighter line spacing

\begin{document}

\title{Annual Temperature in Florida (1901--2000)}
\author{Mikael Minten}
\date{October 2025}
\maketitle

\section*{Introduction}
This analysis examines annual mean temperature data from Key West, Florida, to determine whether temperatures significantly increased during the 20th century. 

\section*{Methods}
We calculated Pearson’s correlation between year and annual temperature. To test if this correlation could occur by chance, we ran a permutation test with 100,000 iterations, shuffling temperature values among years each time and recalculating the correlation. The $p$-value was the proportion of permuted correlations greater than the observed one, testing the null hypothesis of no relationship between year and temperature.

\section*{Results}
The observed correlation coefficient was $r = 0.533$. Of 100,000 random permutations, zero produced a correlation greater than the observed value. Figure~\ref{fig:temp} shows a clear positive trend in annual temperatures across the century.

\begin{figure}[htbp]
\centering
\includegraphics[width=0.6\textwidth]{../data/temperature_plot.pdf}
\caption{Annual mean temperature in Key West, Florida (1901--2000). Blue line shows linear regression.}
\label{fig:temp}
\end{figure}

\section*{Conclusion}
The permutation test provides strong evidence that temperatures in Key West increased significantly during the 20th century. With 100,000 permutations, the probability of observing a correlation of 0.533 or stronger by random chance is zero, allowing us to reject the null hypothesis of no temporal trend.

\end{document}
